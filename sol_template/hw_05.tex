
\IfFileExists{../../../../prefix.tex}{\def\prefixd{../../../..}}{}
\IfFileExists{../../../prefix.tex}{\def\prefixd{../../..}}{}
\IfFileExists{../../prefix.tex}{\def\prefixd{../..}}{}
\IfFileExists{../prefix.tex}{\def\prefixd{..}}{}%
\IfFileExists{prefix.tex} {\def\prefixd{.} }{}%

\input{\prefixd/prefix.tex}
   

\HomeworkStart%
{5} % Homework number
{7} % Week of semester submitted
{1.11} % Version

\SaveIndent%
\HWInstructionsStd{}%

\newcommand{\SqrtMerge}{\AlgorithmI{Sqrt{}Merge}\xspace}%

\newcommand{\Arr}{\texttt{A}}%
\newcommand{\ArrB}{\texttt{B}}%
\newcommand{\Merge}{\Algorithm{Merge}\xspace}%
\newcommand{\sortBlock}{\AlgorithmI{sort{}Block}\xspace}

\begin{questions}[start=13]
    \RestoreIndent%
    \medskip%

    \item \HWProblem{100}{Not a sorting network.}%
    {%
       You are given an array $\Arr[1\ldots n]$, but you can not
       access it directly (or even read the values in it, or compare
       them directly). Instead, you are give some procedures that can
       access the array and do certain operations. Your task is to
       sort the array.

       \begin{questions}
           \item \points{30} You are given a procedure
           $\sortBlock( i)$, which sorts (in increasing order and in
           place), all the elements in $A[ i \ldots, i+u]$, where
           $u \geq 1$ is a prespecified fixed parameter (i.e., $u$ is a
           known fixed number between $1$ and $n$, but it is not under
           your control). For example, for the following array:

           \begin{tabular}{*{7}{|c}|}
             \hline%
             2.71828
             & 18284
             & 59045
             & 23536
             & 02874
             & 7.1352
             & 66249
             \\
             \hline
             \multicolumn{1}{c}{1}
             &
               \multicolumn{1}{c}{2}
             &
               \multicolumn{1}{c}{3}
             &
               \multicolumn{1}{c}{4}
             &
               \multicolumn{1}{c}{5}
             &
               \multicolumn{1}{c}{6}
             & \multicolumn{1}{c}{7}
             \\
           \end{tabular}

           With $u=3$, the call
           $\sortBlock(2)$ would result in:

           \begin{tabular}{*{7}{|c}|}
             \hline%
             2.71828
             & 02874
             & 18284
             & 23536
             & 59045
             & 7.1352
             & 66249
             \\
             \hline
             \multicolumn{1}{c}{1}
             &
               \multicolumn{1}{c}{2}
             &
               \multicolumn{1}{c}{3}
             &
               \multicolumn{1}{c}{4}
             &
               \multicolumn{1}{c}{5}
             &
               \multicolumn{1}{c}{6}
             & \multicolumn{1}{c}{7}
             \\
           \end{tabular}
           
           Describe an algorithm, that uses $O\bigl((n/u)^2\bigr)$
           calls to \sortBlock, and sorts the array.
           
           What is the running time of your algorithm if calling
           \sortBlock takes $O(1)$ time?

           
           \newcommand{\bMerge}{\AlgorithmI{bMerge}\xspace}
           \newcommand{\Copy}{\AlgorithmI{copy}\xspace}
           
           \item \points{50} Congratulations! You just got a better
           sorting primitives $\bMerge, \Copy$, and a work array
           $W[1,\ldots,n ]$.
           
           \begin{compactenumi}
               \item \Copy can copy any block of length at most $u+1$
               between the two arrays (or inside them).
               
               \item $\bMerge$ is weirder.  It takes two blocks $L$
               and $U$ (both with at most $u+1$ elements), treat them
               as a single block, sort the unified block, and writes
               the smaller $|L|$ elements (in sorted order) to $L$,
               and the other (larger) $|U|$ elements in sorted order
               to $U$ (the two blocks do not have to be of the same
               length).
               
               For example, for
               
               \begin{tabular}{*{7}{|c}|}
                 \hline%
                 2.71828
                 & 18284
                 & 59045
                 & 23536
                 & 02874
                 & 7.1352
                 & 66249
                 \\
                 \hline
                 \multicolumn{1}{c}{1}
                 &
                   \multicolumn{1}{c}{2}
                 &
                   \multicolumn{1}{c}{3}
                 &
                   \multicolumn{1}{c}{4}
                 &
                   \multicolumn{1}{c}{5}
                 &
                   \multicolumn{1}{c}{6}
                 & \multicolumn{1}{c}{7}
               \end{tabular}
               
               With $u=1$, the call
               $\bMerge(A[2\ldots3],A[6\ldots 7])$ would result in:
               
               \begin{tabular}{*{7}{|c}|}
                 \hline%
                 2.71828
                 &   7.1352
                 &    18284
                 & 23536
                 & 02874
                 &   59045
                 &   66249
                 \\
                 \hline
                 \multicolumn{1}{c}{1}
                 &
                   \multicolumn{1}{c}{2}
                 &
                   \multicolumn{1}{c}{3}
                 &
                   \multicolumn{1}{c}{4}
                 &
                   \multicolumn{1}{c}{5}
                 &
                   \multicolumn{1}{c}{6}
                 & \multicolumn{1}{c}{7}
               \end{tabular}
               
               \bMerge also returns the number of elements in original
               block $L$ that are still in $L$ after the operation is
               completed. In this example, since $18284$ was in $L$
               before the operation, and it is in $L$ after the
               operation is completed, the returned value would be
               $1$.
           \end{compactenumi}
           
           Note  that the blocks given to \bMerge can not overlap.

           Assume that $A[1\ldots n/2]$ and $A[ n/2+1, \ldots n]$ are
           already sorted ($n$ is even).  Describe an algorithm that
           performs a minimal total number of calls to \sortBlock,
           \Copy and \bMerge, and sorts the array $A$. What is the
           running time of your algorithm if calling \sortBlock, \Copy
           and \bMerge takes $O(1)$ time?  (Prove your bound.)

           

           \item \points{20} Building on (B) and expanding on it, describe a
           sorting algorithm using these primitives that sort the
           given array $A$ (that is initially not sorted).  What is
           the running time of your algorithm if calling \sortBlock,
           \Copy and \bMerge takes $O(1)$ time? Naturally, the faster
           the better (Prove your bound).
           
       \end{questions}
    } {}{}{}{}%
    \medskip%
    

    \item \HWProblem{100}{Not a sorting question.}%
    {%
       
       Consider an array $A[\range{0}{n-1}]$ with $n$ distinct
       elements. Each element is an $\ell$ bit string representing a
       natural number between $0$ and $2^\ell -1$ for some $\ell
       >0$. The only way to access any element of $A$ is to use the
       function $\AlgorithmI{FetchBit}(i,j)$ that returns the $j$\th
       bit of $A[i]$ in $O(1)$ time.
       \begin{questions}
           \item \points{20} Suppose $n=2^\ell-1$, i.e. exactly one of
           the $\ell$-bit strings does not appear in $A$. Describe an
           algorithm to find the missing bit string in $A$ using
           $\Theta(n\log n)$ calls to $\AlgorithmI{FetchBit}$ without
           converting any of the strings to natural numbers.
           
           \item \points{40} Suppose $n=2^\ell-1$. Describe an
           algorithm to find the missing bit string in $A$ using only
           $O(n)$ calls to $\AlgorithmI{FetchBit}$.
           
           \item \points{40} Suppose $n=2^\ell-k$, i.e. exactly $k$ of
           the $l-$bit strings do not appear in $A$.  Describe an
           algorithm to find the $k$ missing bit strings in $A$ using
           only $O(n\log k)$ calls to $\AlgorithmI{FetchBit}$.
           
       \end{questions}
    }{}{}{}{}%
    \medskip%
    
    \item \HWProblem{100}{Don't want to walk too much.}{%
       
       You are given a set of $n$ distinct points on a line with
       x-coordinates $x_1,x_2,\,..\, x_n$. The points are not sorted
       and their values are stored in an array $X$ where $X[i] =
       x_i$. Each point is associated with a positive weight $w_i$
       such that $\sum{w_i} = 1$. The weights are also stored in an
       array $W$ where $W[i] = w_i$. Our goal is to find $x_j$ that
       minimizes the weighted distance given by:
       $ \sum_i{w_i|x_j-x_i|}$.
       
       \begin{questions}
           \item \points{10} (Easy.) Show that if all the weights are
           equal, $x_j$ is the median of $X$.
           
           \item \points{20} In general, show that $x_j$ is the point
           that satisfies the following property:
           
           \begin{equation*}
               \sum_{x_i<x_j}{w_i} < \frac{1}{2}
               \qquad\text{and}\qquad
               \sum_{x_i>x_j}{w_i} \le \frac{1}{2}.
           \end{equation*}
           
           \item \points{20} Given, $X$ and $W$, describe in few lines
           an algorithm to find $x_j$. What is the running time of
           your algorithm.
           
		   
	   \item \points{50} Given, $X$ and $W$, describe an algorithm
           to find $x_j$ in $O(n)$ time. Prove the bound on the
           running time of your algorithm.

       \end{questions}
    }{}{}{}%

\end{questions}

\HomeworkEnd{}

