\ifx\AllTogetherMode\undefined%
\IfFileExists{../prefix.tex}%
{\input{../prefix}}%
{\input{prefix}}%
\fi

\HomeworkStart%
{3} % Homework number
{4} % Week of semester submitted
{1.21} % Version

\SaveIndent%

\begin{questions}[start=7]
    \RestoreIndent%
    
    \medskip%
    \item \HWProblem{100}{Draw me a giraffe.}{%
       For each of the following languages in \ref{Q1a}--\ref{Q1c},
       draw an \NFA that accepts them. Your automata should have a
       small number of states. Provide a short explanation of your
       solution, if needed.
       \begin{questions}%[(A)]
           \item\label{Q1a} \points{25} %
           All strings in $\{ \T{0}, \T{1}, \T2 \}^*$ such that at
           least one of the symbols $\T0$, $\T1$, or $\T2$ occurs at
           most 4 times.  (Example: $\T{1200201220210}$ is in the
           language, since $\T1$ occurs $3$ times.)

           \item \points{25} %
           $\big((\T0\T1)^*(\T1\T0)^* +
           \T{00}\big)^*\cdot(\T1+\T{00}+\T\eps)\cdot (\T{11})^*$.

           \item\label{Q1c} \points{25} %
           All strings in $\{\T{0},\T{1}\}^*$ such that the last
           symbol is the same as the third last symbol.  (Example:
           $\T{1100101}$ is in the language, since the last and the
           third last symbol are $\T1$.)
						
           \item \points{25} %
           Use the power-set construction (also called subset
           construction) to convert your \NFA from \ref{Q1c} to a
           \DFA.  You may omit unreachable states.

  
       \end{questions}
       
    }{}{}{}

    %%%%%%%%%%%%%%%%%%%%%%%%%%%%%%%%%%%%%%%%%%%%%%%%%%%%%%%%%%%%%%%%%%%%
    %%%%%%%%%%%%%%%%%%%%%%%%%%%%%%%%%%%%%%%%%%%%%%%%%%%%%%%%%%%%%%%%%%%%
    %%%%%%%%%%%%%%%%%%%%%%%%%%%%%%%%%%%%%%%%%%%%%%%%%%%%%%%%%%%%%%%%%%%%
    \bigskip%
    \item \HWProblem{100}{Fun with parity.}{%
       \newcommand{\EVEN}{\textit{even}_0}
       \newcommand{\ODD}{\textit{odd}_0}
    
       Given $L\subseteq\{\T0,\T1\}^*$, define $\EVEN(L)$ to be the
       set of all strings in $\{\T0,\T1\}^*$ that can be obtained by
       taking a string in $L$ and inserting an even number of \T0's
       (anywhere in the string).  Similarly, define $\ODD(L)$ to be
       the set of all strings $x$ in $\{\T0,\T1\}^*$ that can be
       obtained by taking a string in $L$ and inserting an odd number
       of $\T0$'s.
       
       
       (Example: if $\T{01101}\in L$, then
       $\T{01010000100}\in\EVEN(L)$.)
       
       (Another example: if $L$ is $\T1^*$, then $\EVEN(L)$ can be
       described by the regular expression
       $(\T1^*\T0\T1^*\T0)^*\T1^*$.)
       
       The purpose of this question is to show that if
       $L\subseteq\{\T0,\T1\}^*$ is regular, then $\EVEN(L)$ and
       $\ODD(L)$ are regular.
       
       \begin{questions}
           \item \points{30} %
           For each of the base cases of regular expressions
           $\emptyset$, $\T\eps$, $\T0$, and $\T1$, give regular
           expressions for $\EVEN(L(r))$ and $\ODD(L(r))$.

           \item \points{60} %
           Given regular expressions for $e_j=\EVEN(L(r_j))$ and
           $o_j=\ODD(L(r_j))$, for $j\in\{1,2\}$, give regular
           expressions for
           \begin{compactenumi}[labelwidth=0.4cm,%
               itemsep=-0.1cm,%
               itemindent=0.4cm,align=parleft]
               \item $\EVEN(L(r_1+r_2))$
               \item $\ODD(L(r_1+r_2))$
               \item $\EVEN(L(r_1r_2))$
               \item $\ODD(L(r_1r_2))$
               \item $\EVEN(L(r_1^*))$
               \item $\ODD(L(r_1^*))$
           \end{compactenumi}
           Give brief justification of correctness for each of the
           above.

           \item \points{10} %
           Using the above, describe (shortly) a recursive algorithm
           that given a regular expression $r$, outputs a regular
           expression for $\EVEN(L(r))$ (similarly describe the
           algorithm for computing $\ODD(L(r))$).
       \end{questions}
    }{}{}{}
    
    
    %%%%%%%%%%%%%%%%%%%%%%%%%%%%%%%%%%%%%%%%%%%%%%%%%%%%%%%%%%%%%%%%%%%% 
    %%%%%%%%%%%%%%%%%%%%%%%%%%%%%%%%%%%%%%%%%%%%%%%%%%%%%%%%%%%%%%%%%%%% 
    %%%%%%%%%%%%%%%%%%%%%%%%%%%%%%%%%%%%%%%%%%%%%%%%%%%%%%%%%%%%%%%%%%%% 
    \bigskip%
    \item \HWProblem{100}{``+1''.}%
    {%
       \newcommand{\INC}{\mathrm{INC}}%
       \newcommand{\binary}{\textrm{binary}}%
       Let $\binary(i)$ denote the binary representation of a positive
       integer $i$.  (Note that the string $\binary(i)$ must start
       with a $\T1$.)
       
       Given a language $L\subseteq \{\T0,\T1\}^*$, define
       $\INC(L)=\{\binary(i+1) \mid \binary(i)\in L\}$. For the time
       being assume that $L$ does not contain any string of $\T1^*$.
       
       (Example: for $L=\{\T{100},\T{101011},\T{1101} \}$, we have
       $\INC(L)=\{\T{101},\T{101100},\T{1110} \}$.)
       
       \begin{questions}
           \item \points{30} Given a \DFA
           $M = (Q, \Sigma, \delta, s, A)$ for $L$, describe
           \textbf{informally} (in a few sentences) how to construct
           an \NFA $M_w$ for $\INC(L)$.
           
           
           \item \itemlab{9:b} \points{30} Given a \DFA
           $M = (Q, \Sigma, \delta, s, A)$ for $L$, describe
           \textbf{formally} how to construct an \NFA $M'$ for
           $\INC(L)$.
           
           
           
           \item \points{30} Prove formally the correctness of your
           construction from \itemref{9:b}. That is, prove that
           $\INC(L) = L(M')$.
           
           \item \points{10} Describe formally how to modify the
           construction of $M'$ from above, to handle that general
           case (without the above assumption) that $L$ might also
           contain strings of the form $\T1^*$. You do not need to
           provide a proof of correctness of the new automata.
       \end{questions}
    }{}{}

\end{questions}


\HomeworkEnd{}



    %%%%%%%%%%%%%%%%%%%%%%%%%%%%%%%%%%%%%%%%%%%%%%%%%%%%%%%%%%%%%%%%%%%%
    %%%%%%%%%%%%%%%%%%%%%%%%%%%%%%%%%%%%%%%%%%%%%%%%%%%%%%%%%%%%%%%%%%%%
    

\bigskip%
\item \OLDHWProblem{100}{Codes.}{%
   Let $\Sigma$ be finite alphabet. A \emphi{code} is a mapping
   $f:\Sigma \rightarrow \brc{\T0,\T1}^+$. For example, if
   $\Sigma = \brc{ \T{a}, \T{b},\T{c}}$, a code $f$ might be
   $f( \T{a}) = \T{00010}$, $f( \T{b}) = \T{000}$, and
   $f( \T{c}) = \T{1}$. (To simplify things, we assume that
   $f(a) \neq \eps$, for any character $a \in \Sigma$.)

       
   For a string $w_1 w_2 \cdots w_m \in \Sigma^*$, we define
   $f(w) = f(w_1) f(w_2) \cdots f(w_m)$. In the above code, we have
   \begin{align*}
     f( \T{abcba})%
     =%
     \T{00010}%
     \Cdot%
     \T{000} %
     \Cdot%
     \T{1}%
     \Cdot%
     \T{000} %
     \Cdot%
     \T{00010}.
     =
     \T{00010}%
     \T{000} %
     \T{1}%
     \T{000} %
     \T{00010}.
   \end{align*}

   \medskip%
   \begin{questions}%[(A)]
       \item \points{10} Let $L$ be the language of the following \DFA
       $M$. What is $L$?

       \centerline{\includegraphics[scale=0.35]%
          {\File{figs/dfa}}}%

           
       \item \points{20} Working directly on the \DFA $M$ from (A)
       construct an \NFA for the language $f(L)$. Here
       $f(L) = \Set{f(w)}{w \in L}$ is the \emph{code language}. Where
       $f$ is code from the above example.


       \medskip%
       \item \points{30} Let $L \subseteq \Sigma^*$ be am arbitrary
       regular language. Prove that the encoded language
       $f(L) = \Set{f(w)}{w \in L}$ is regular.

       \medskip%
       Specifically, given a \DFA $M = (Q, \Sigma, \delta, s, A)$ for
       $L$, describe how to build an \NFA $M'$ for $f(L)$. Give an
       upper bound on the number of states of $M'$.
           
       (I.e., You need to prove the correctness of your construction
       -- that the language of the constructed \NFA is indeed the
       desired language $f(L)$.)

       (Rubric: Half the credit is for a correct construction, and the
       other half is for a correct proof of correctness.)

           
           
           
       \medskip%
       \item \points{40} Let $L \subseteq \brc{\T0,\T1}^*$ be a
       regular language. Consider the decoded language
       $L_f = \Set{ w \in \Sigma^*}{f(w) \in L}$.

       Prove that $L_f$ is a regular language. As above, given a \DFA
       $M$ for $L$, describe how to construct an \NFA for $L_f$.

       (Rubric: Half the credit is for a correct construction, and the
       other half is for a correct proof of correctness.)

           
           
   \end{questions}
       
}{}{}{}
    

