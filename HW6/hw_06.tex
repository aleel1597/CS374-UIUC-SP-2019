\ifx\AllTogetherMode\undefined%
\IfFileExists{../prefix.tex}%
{\input{../prefix}}%
{\input{prefix}}%
\fi

\HomeworkStart%
{6} % Homework number
{8} % Week of semester submitted
{1.1} % Version

\SaveIndent%
\HWInstructions{%
   \textbf{Submission instructions as in previous %
      \underline{%
         \href{https://courses.engr.illinois.edu/cs374/fa2017/hw/hw_00.pdf}{homeworks}%
      }.}%

\medskip%
\hrule %
\medskip
}

\newcommand{\EE}{\mathcal{E}}%
\begin{questions}[start=16]
    \RestoreIndent%
    \medskip%
    
    
    \item \HWProblem{100}{Simplifying data.}%
    {%
       A \emph{$k$-step function} is a function of the form
       \[ f(x) = b_i \qquad\mbox{if
              $a_i\le x < a_{i+1}\ \ (i=0,\ldots,k-1)$} \] for some
       $-\infty=a_0<a_1<\cdots<a_{k-1}<a_k=\infty$ and some
       $b_0,b_1,\ldots,b_{k-1}$.
       
       We are given $n$ data points
       $p_1=(x_1,y_1),\ldots,p_n=(x_n,y_n)$ and a number $k$ between 1
       and $n$.  Our objective is to find a $k$-step function $f$ such
       that $f(x_i)\ge y_i$ for all $i\in\{1,\ldots,n\}$, while
       minimizing the total ``error'' $\sum_{i=1}^n (f(x_i)-y_i)$
       (this is the total length of the red vertical segments in the
       figure below).
       
       \begin{figure}[h]
           \centering \includegraphics[scale=1]{figs/histogram.pdf}
       \end{figure}
       
       \begin{questions}
           \item \points{70} Describe an algorithm, as fast as
           possible, that computes the minimum total error of the
           optimal $k$-step function. Bound the running time of your
           algorithm as a function of $n$ and $k$.
           
           [Note: in dynamic programming questions such as this, first
           give a clear English description of the function you are
           trying to evaluate, and how to call your function to get
           the final answer, then provide a recursive formula for
           evaluating the function (including base cases).  If a
           correct evaluation order is specified clearly, iterative
           pseudocode is not required.]
           
           
           \item \points{30} Describe how to modify your algorithm in
           (A) so that it computes the optimal $k$-step function
           itself.
           
       \end{questions}
    }
    
    
    \item \HWProblem{100}{Closest subsequence}%
    { Define the \emph{$L_1$-distance} between two sequences of real
       numbers $\langle a_1,\ldots,a_m\rangle$ and
       $\langle b_1,\ldots,b_m\rangle$ to be
       $|a_1-b_1|+\cdots+|a_m-b_m|$.
       
       Consider the following problem: given two sequences of real
       numbers $A=\langle a_1,\ldots,a_m\rangle$ and
       $B=\langle b_1,\ldots,b_n\rangle$ with $m\le n$, find a
       subsequence of $B$ of length $m$ that minimizes its
       $L_1$-distance to $A$.
       
       \begin{questions}
           \item \points{70} Describe an algorithm, as fast as
           possible, that computes the $L_1$-distance of the optimal
           subsequence of $B$ to $A$. Bound the running time of your
           algorithm as a function of $m$ and $n$.
           
           \item \points{30} Describe how to modify your algorithm in
           (A) so that it computes the optimal subsequence itself.
       \end{questions}
    }
    
    
    \newpage
    \item \HWProblem{100}{Fold it!}%
    { We are given a ``chain'' with $n$ links of lengths
       $a_1,\ldots, a_n$, where each $a_i$ is a positive integer.  We
       are also given a positive integer $L$.  We want to determine if
       it is possible to ``fold'' the chain (in one dimension) so that
       the length of the folded chain is at most $L$.  More formally,
       we want to decide whether there exists $t\in [0,L]$ and
       $s_1,\ldots,s_n\in\{-1,+1\}$ such that
       $t + \sum_{i=1}^j s_i a_i\in [0,L]$ for all
       $j\in\{0,\ldots, n\}$.  (Here, $t$ denotes the starting
       position, and $s_i=\pm 1$ depending on whether we turn
       rightward or leftward for the $i$\th link.)
       
       Example: for $a_1=5$, $a_2=1$, $a_3=7$, $a_4=2$, $a_5=8$, and
       $L=9$, a solution is shown below.
       
       \begin{figure}[h]
           \centering \includegraphics{figs/zig_zag}
       \end{figure}
       
       
       \begin{questions}
           \item \points{70} Provide an $O(nL)$-time algorithm to
           decide whether a solution exists.  (Argue why the stated
           running time is correct.)
           
           Partial credit would be given to slower solutions with
           running time $O(nL^2)$ or $O(n L^3)$.

           
           
           \item \points{30} Using (A) as a subroutine, describe an
           algorithm (as fast as possible) to find the minimum length
           $L$ such that a valid folding exists. 
           $L^*$ of the best folding.  What is the running time of
           your algorithm as a function of $n$ and $L^*$?
       \end{questions}
    }
    
    
    
    
\end{questions}

\HomeworkEnd{}

